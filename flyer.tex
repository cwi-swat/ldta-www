\documentclass[10pt]{article}
\usepackage{graphicx,url}
\usepackage[margin=0.75in]{geometry}
\usepackage[compact,small]{titlesec}

\pagestyle{empty}
%\usepackage{changebar}

\newenvironment{itemlist}{
                         \begin{list}{{$\bullet$}} 
                         {\setlength{\itemsep}{0pt} 
                         \setlength{\topsep}{0pt} 
                         \setlength{\partopsep}{2pt}
                         \setlength{\leftmargin}{6pt} 
                         \setlength{\labelwidth}{8pt} 
                         \setlength{\labelsep}{2pt} 
                         \setlength{\listparindent}{0pt}
                         \setlength{\parsep}{0cm}
                         \usecounter{enumi} 
                         \setlength{\parskip}{0.2cm}}}{\end{list} }

\newenvironment{inditemlist}{
                         \begin{list}{{$\bullet$}} 
                         {\setlength{\itemsep}{0pt} 
                         \setlength{\topsep}{0pt} 
                         \setlength{\partopsep}{2pt}
                         \setlength{\leftmargin}{12pt} 
                         \setlength{\labelwidth}{8pt} 
                         \setlength{\labelsep}{2pt} 
                         \setlength{\listparindent}{0pt}
                         \setlength{\parsep}{0.07cm}
                         \usecounter{enumi} 
                         \setlength{\parskip}{0.2cm}}}{\end{list}\vspace{2mm} }

\begin{document}

\noindent
\begin{minipage}[b][1.6in][t]{1in}
\includegraphics[width=1.0\columnwidth]{images/etapsLogo.png}

\end{minipage}
\begin{minipage}[b][1.6in][t]{5in}
\begin{center}
Call for Papers and Tool Challenge Submissions for

\vspace{2mm}

{\Large $11^{th}$ Workshop on \\
Language Descriptions, Tools and Applications}

\vspace{3mm}

{\Large LDTA 2011}

\vspace{3mm}

March 26 \& 27, 2010 in Saarbr\"ucken, Germany, an ETAPS workshop

\vspace{2mm}

\url{http://ldta.info}
\end{center}
\end{minipage}
\begin{minipage}[b][1.6in][t]{1in}
\includegraphics[width=1.0\columnwidth]{images/Tools.png}
\vspace{-8mm}
\begin{center}
Tool \\
Challenge
\end{center}
\end{minipage}

\newcommand{\pcmem}[3]{#1}%, #2, #3}
\newcommand{\pcchair}[4]{#1, (co-chair)\\ #4}

\noindent
\begin{minipage}[t]{1.9in}
\textbf{Important Dates:}
\begin{itemlist}
\item Abstract submission: \\ Dec. 15, 2010
\item Full paper submission: \\ Dec. 22, 2010
\item Author notification: \\ Feb. 4, 2011
\item Tool challenge submission: \\ Mar. 05, 2011
%\item Camera-ready papers: TBD 
\item LDTA Workshop: \\ Mar. 26-27, 2011 ;
\end{itemlist}

\medskip
\noindent
\textbf{Invited Speaker:}
\begin{itemlist}
\item Rinus Plasmeijer, Radboud \\ University Nijmegen,\\ The Netherlands
\end{itemlist}

\medskip
\noindent
\textbf{Program\\ Committee:}
\begin{itemlist}
\item \pcmem{Emilie Balland}{INRIA}{France}
\item \pcmem{Paulo Borba}{Federal University of Pernambuco}{Brazil}
\item \pcmem{John Boyland}{University of Wisconsin - Milwaukee}{USA}
\item \pcchair{Claus Brabrand}{IT University of Copenhagen}{Denmark}{brabrand@itu.dk}
\item \pcmem{Kyung-Goo Doh}{Hanyang University, Ansan}{South Korea}
\item \pcmem{Rob Economopoulos}{University of Southampton}{UK}
\item \pcmem{Nigel Horspool}{University of Victoria}{Canada}
\item \pcmem{Roberto Ierusalimschy}{Pontifícia Universidade Católica do Rio de Janeiro}{Brazil}
\item \pcmem{Johan Jeuring}{Utrecht University}{The Netherlands}
\item \pcmem{Shane Markstrum}{Bucknell University}{USA}
\item \pcmem{Jo\~ao Saraiva}{Universidade do Minho}{Portugal}
\item \pcmem{Eli Tilevich}{Virginia Tech}{USA}
\item \pcchair{Eric Van Wyk}{University of Minnesota}{USA}{evw@cs.umn.edu}
\item \pcmem{Eelco Visser}{Delft University of Technology}{The
    Netherlands}
% Check that there is room for 20 names in the column.
%\item \pcmem{Eelco Visser}{Delft University of Technology}{The Netherlands}
%\item \pcmem{Eelco Visser}{Delft University of Technology}{The Netherlands}
%\item \pcmem{Eelco Visser}{Delft University of Technology}{The Netherlands}
%\item \pcmem{Eelco Visser}{Delft University of Technology}{The Netherlands}
%\item \pcmem{Eelco Visser}{Delft University of Technology}{The Netherlands}
%\item \pcmem{Eelco Visser}{Delft University of Technology}{The Netherlands}
\item ... more to be added ...
\end{itemlist}

\medskip
\noindent
\textbf{Organizing\\ Committee:}
\begin{itemlist}
\item \pcmem{Emilie Balland}{INRIA}{France}
\item \pcmem{Rob Economopoulos}{University of Southampton}{UK}
\end{itemlist}
\end{minipage}
%\begin{minipage}[t]{0.1in}
%\end{minipage}
\begin{minipage}[t]{5.1in}
LDTA is an application and tool-oriented workshop focused on
grammarware, software based on grammars in some form.  Grammarware
applications are typically language processing applications and
traditional examples include parsers, program analyzers, optimizers
and translators.  A primary focus of LDTA is grammarware that is
generated from high-level grammar-centric specifications and thus
submissions on parser generation, attribute grammar systems,
term/graph rewriting systems, and other grammar-related
meta-programming tools, techniques, and formalisms are encouraged.

LDTA is also a forum in which theory is put to the test, in many cases
on real-world software engineering challenges.  Thus, LDTA also
solicits papers on the application of grammarware to areas including,
but not limited to, the following:
\begin{inditemlist}
\item Program analysis, transformation, generation, and verification
\item Implementation of domain specific languages
\item Reverse engineering and re-engineering
\item Refactoring and other source-to-source transformations
\item Language definition and language prototyping
\item Debugging, profiling, IDE support, and testing
\end{inditemlist}

Note that LDTA is a well-established workshop similar to other
conferences on (programming) language engineering topics such as SLE
and GPCE, but is solely focused on grammarware.

\medskip
\textbf{Paper Submission:}
LDTA solicits papers in the following categories:
\begin{inditemlist}
\item research papers: original research results within the scope of LDTA
  with a clear motivation, description, analysis, and evaluation, 15 pages.

\item short research papers: new ideas that have not been
  completely fleshed out.  As a workshop, LDTA strongly encourages
  these types of submissions. 6 pages.

\item experience report papers: description of the use of a grammarware
  tool or technique to solve a non-trivial applied problem, 15 pages.

\item tool demo papers: discussion of an innovative tool or technique,
  10 pages.
\end{inditemlist}
The final version of the accepted papers will,
pending approval, be published in the ACM Digital Library.  Authors of
the best papers may be invited to write a journal version for a
special journal issue.
%
The authors of each submission are required to give a presentation at
LDTA 2011 and tool demonstration paper presentations are intended to
include a significant live, interactive demonstration.
%
Full paper submission instructions available at \url{http://ldta.info}.

\medskip
\textbf{Tool Challenge:}
This year LDTA will also be putting theory, as well as techniques and
tools, to the test in a new way - in the LDTA Tool Challenge.  Tool
developers are invited to participate in the Challenge by developing
solutions to a range of language processing tasks over a simple but
evolving set of imperative programming languages.  Tool challenge
participants will present highlights of their solution during a
special session of the workshop and contribute to a joint paper on the
Tool Challenge and proposed solutions to be co-authored by all
participants after the workshop.  
%
A full description of the Challenge is available at
\url{http://ldta.info}.  It includes the set of
problems and suggestions for discussing the success of a specific
tool or technique.






\end{minipage}


\end{document}
